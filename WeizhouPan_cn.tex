%% start of file `template.tex'.
%% Copyright 2006-2010 Xavier Danaux (xdanaux@gmail.com).
%
% This work may be distributed and/or modified under the
% conditions of the LaTeX Project Public License version 1.3c,
% available at http://www.latex-project.org/lppl/.

% Version: 20110122-4


\documentclass[11pt,a4paper,nolmodern]{moderncv}

\usepackage{WeizhouPan}
\moderncvtheme[blue]{classic}

\usepackage[english]{babel}
\linespread{0.95}

\setCJKmainfont[BoldFont={Adobe Heiti Std},ItalicFont={Adobe Kaiti
  Std}]{Adobe Song Std}
\setCJKsansfont{YaHei Consolas Hybrid}
\setCJKmonofont{Adobe Kaiti Std}

\title{Weizhou Pan, 硕士研究生}
\address{RM 504~(510631)}{华南师范大学~计算机学院}
\extrainfo{\octocat~\url{http://github.com/wzpan}\\
  \linkedin~\httplink{www.linkedin.com/in/hahack}%
}
\myquote{Keep a simple and stupid mind.}{}

%\extrainfo{\octocat~\url{http://github.com/wzpan}}

%\nopagenumbers{}                             % uncomment to suppress automatic page numbering for CVs longer than one page
%----------------------------------------------------------------------------------
%            content
%----------------------------------------------------------------------------------
\begin{document}

\hyphenpenalty=10000
\maketitle

\section{\hei 技能}

\subsection{\hei 系统}
\cvline{操作系统}{GNU/Linux(ArchLinux, Ubuntu, RedHat, Linux Deepin, Fedora), Windows}
\cvline{桌面环境}{Awesome, xfce, KDE, Gnome2, Gnome2, Gnome3, Unity, Cinnamon, Mate}

\subsection{\hei 开发}
\cvcomputer{语言}{Python, Shell/Bash, C, Java, Pascal, Matlab}
	   {格式}{Markdown, Org-mode, Textile, reStructuredText, XML, YAML/JSon}

\cvcomputer{Web}{HTML, CSS, Mustache, Ruhoh, Wordpress, Twitter-Bootstrap, Web.py}
           {数据库}{MySQL, SQL Server}

\cvcomputer{编辑器}{Emacs, Vim, Subl, Visual Studio, Eclipse}
           {文档}{Doxygen, CHM}
           
\cvcomputer{方法}{设计模式, 面向对象, 测试驱动}
           {版本控制}{Git, SVN, GitHub, Google Code}


\subsection{\hei 工具}
\cvcomputer{办公}{Microsoft Office, WPS, OpenOffice/LibreOffice}
           {图形}{TikZ, Graphviz, Ditta, Gimp, Inkscape, Photoshop, Fireworks, CorelDraw, Visio, yEd, Calligra Flow}

\cvcomputer{效率}{Mind Map, GTD, Evernote, Dropbox, Org-mode, AutoHotKey, Snippets, Gist}
           {Wikis}{MoinMoin, MediaWiki, Github Wiki}
\cvcomputer{排版}{\TeX{}, \LaTeX{}, \XeTeX{}}
           {键盘布局}{QWERTY, Dvorak}

%\devnotes{Developer}{Contributor}

\section{\hei 履历}
\subsection{\hei 科研实践}

% Center labels and use "Since"
%\tltextstart[base]{\scriptsize}
%\tltextend[base]{\scriptsize}
%\tlsince{Since~}

\tlcventry{2012}{2013}{\href{http://www.siat.ac.cn/}{中国科学院深圳先进技术研究
    院}}{客座学生}{}{}
{
\begin{tightitemize}%
 \item 与\href{http://web.siat.ac.cn/~baoquan/}{陈宝权教授},
   \href{http://www.math.tau.ac.il/~dcor/}{Daniel Cohen-Or教授},
   \href{http://graphics.uni-konstanz.de/mitarbeiter/deussen.php}{Oliver
     Deaussen教授}, \href{http://www.idav.ucdavis.edu/~asharf/}{Andrei Sharf博士}等有着
   高效合作;
 \item 研究材质演变、形状检索、运动数据聚类等图形学领域课题;
 \item 为\href{http://web.siat.ac.cn/~yangyan/}{李扬彦}博士的科研成果制作\href{http://hahack.com/life/video-released}{展示视频};
  \item 连续获得两个季度的客座学生奖学金.
 \end{tightitemize}}

\tlcventry{2011}{2012}{手势识别及红外感应在虚拟手术场景人机交互的研究和应用}{核心成员}{}{}%
  {
\begin{itemize}
 \item \textbf{课题性质}:广东省产学研结合项目《数字医学在腹部外科仿真手术的研究和开发》 子课题.
 \item 研究内容:采用立体视觉和红外捕捉技术实现了一支仿真手术刀,并结合基于BP神经网络的手势识别指令,构建出结合虚拟手术操作和场景漫游功能的沉浸式手术环境;
 \item \textbf{个人职责}:研究基于任天堂 Wii Remote 的红外感应技术;基于 OpenGL 的虚拟手
   术刀的建模;基于 BP 人工神经网络的手势识别;论文撰
   写;
 \item \textbf{项目成果}:发表两篇核心期刊论文;获得2012年华南师范大学软件设计大赛一等奖,2011年第十二届“挑战杯”全国大学生课外学术科技作品竞赛银奖等奖项.
\end{itemize}}

\tlcventry{2010}{2011}{\href{http://www.siat.ac.cn/}{中国科学院深圳先
    进技术研究院}}{客座学生}{}{}
{
\begin{tightitemize}%
 \item 开发二维百度地图控件;
 \item 开发离线地图数据加密模块;
 \item 百维达公司(中国科学院产学研孵化企业)HR助理;
 \item 文献调研.
 \end{tightitemize}}

\tlcventry{2009}{2010}{基于 MLTS 技术的仿生歌唱系统}{负责人}{}{}%
  {
\begin{itemize}
\item \textbf{课题性质}:国家级大学生创新性实验计划项目.
\item \textbf{研究内容}: 提出了一种实现计算机学习并演唱歌曲的系统。系统运用敲击定位法定
  位发音时刻,然后利用 Daubechies 小波变换和快速傅里叶变换计算出对应的基频,采用语音合成技术输出声音.
\item \textbf{个人职责}:系统主要开发者,论文撰写人.
 \item \textbf{项目成果}:发表一篇核心期刊论文;获得华南师范大学软件设计大赛二等奖.
\end{itemize}}

\subsection{\hei 社会实践}

\tlcventry{2011}{0}{\href{http://www.gzlug.org}{广州Linux用户组(GZLUG)}}{成员}{}{}{
\begin{tightitemize}%
 \item \href{http://www.gzlug.org/?p=641}{2012 广州地区软件自由日庆祝活动}组织者;
 \item GZLUG 邮件列表活跃用户.
\end{tightitemize}}

\tlcventry{2008}{2009}{华南师范大学校学生会网络部}{部长}{}{}
{
  \begin{tightitemize}%
    \item 校学生会第32届第3任网络部部长;
    \item 参与组织“女生节”、“宿舍文化节”、当代大学生论坛、课件制作大赛”、师
      范技能大赛等多个校级活动;
    \item 年底被推选为校学生会系统“优秀学生干部”.
 \end{tightitemize}}

\tlcventry{2007}{2008}{华南师范大学计算机学院07级5班}{班长}{}{}
{
  \begin{tightitemize}%
    \item 亲自策划并带头组织“班级文化节”;
    \item 年底被推选为校学生会系统“优秀学生干部”.
    \end{tightitemize}}

\subsection{\hei 论文和专利}
{
  \tldatelabelcventry{2013}{April 2013}{基于人工神经网络的百度地图坐标解密方法
  }{计算机工程与应用}{第一作者}{已发表}{}
  \setlength{\parskip}{-20pt}
  \tldatelabelcventry{2013}{April 2013}{沉浸式手术环境的研究与实现}{计算机仿
    真}{第二作者}{已录用}{}  
  \tldatelabelcventry{2013}{February 2013}{红外光虚拟手术刀的研究与实现}{计算机
    系统应用}{第一作者}{已录用}{}
  \tldatelabelcventry{2012}{April 2012}{一种基于小波和快速傅里叶变换的学习型歌唱系统
}{计算机工程与应用}{第一作者}{已发表}{}
  \tldatelabelcventry{2013}{Augest 2013}{一种同步播放多媒体文件的方法及系统}{发
    明}{第一申请人}{申请中}{}
  \tldatelabelcventry{2012}{December 2012}{基于Clifford代数在十二方向上三维分割
    腹部血管的应用实例}{发明}{合作申请人}{申请中}{}  
}

\subsection{\hei 奖项}

% Restore normal labels
%\tltext{\scriptsize}

{
\tldatelabelcventry{2012}{June 2012}{华南师范大学创新奖学金一等奖}{校级}{}{}{}

\setlength{\parskip}{-20pt}

\tldatelabelcventry{2011}{October 2011}{第十二届“挑战杯”全国大学生课外学术科技
  作品竞赛银奖}{国家级}{}{}{}

\tldatelabelcventry{2011}{June 2011}{华南师范大学优秀毕业生}{校级}{}{}{}

\tldatelabelcventry{2011}{May 2011}{华南师范大学优秀本科毕业论文}{校级}{}{}{}

\tldatelabelcventry{2011}{May 2011}{第十一届“挑战杯”广东大学生课外学术科技作品
  竞赛特等奖}{省级}{}{}{}

\tldatelabelcventry{2011}{April 2011}{华南师范大学一等奖学金 $\times$ 2}{校级}{}{}{}

\tldatelabelcventry{2010}{July 2010}{华南师范大学二等奖学金 $\times$ 2}{校级}{}{}{}

}

\section{\hei 教育背景}

\tlcventry{2011}{0}{\href{http://www.scnu.edu.cn}{华南师范大
    学}~\href{http://portal.scnu.edu.cn/wps/portal/cs}{计算机学院}}{研究生}{}{}{华南师范大
  学计算机学院2011届计算机应用技术硕士研究生}

\tlcventry{2007}{2011}{\href{http://www.scnu.edu.cn}{华南师范大
    学}~\href{http://portal.scnu.edu.cn/wps/portal/cs}{计算机学院}}{本科生}{}{}{华南师范
  大学计算机学院2007届网络工程系学生}


\section{\hei 英语水平}
\cvlanguage{英语6级}{513}{听力: 156; 阅读: 199; 综合技能: 64;写作: 94}


\end{document}
