%% start of file `template.tex'.
%% Copyright 2006-2010 Xavier Danaux (xdanaux@gmail.com).
%
% This work may be distributed and/or modified under the
% conditions of the LaTeX Project Public License version 1.3c,
% available at http://www.latex-project.org/lppl/.

% Version: 20110122-4


\documentclass[11pt,a4paper,nolmodern]{moderncv}

\usepackage{WeizhouPan}
\usepackage{amsmath, amssymb, latexsym}
\usepackage[english]{babel}
\linespread{0.9}

\setCJKmainfont[BoldFont={Adobe Heiti Std},ItalicFont={Adobe Kaiti
  Std}]{Adobe Song Std}
\setCJKsansfont{YaHei Consolas Hybrid}
\setCJKmonofont{Adobe Kaiti Std}

\title{Weizhou Pan, 高级工程师}
\address{\href{http://tencent.com}{腾讯公司}}{}
\extrainfo{\octocat~\url{http://github.com/wzpan}\\
\linkedin~\httplink{www.linkedin.com/in/hahack}%
}
%\myquote{关注人工智能技术}{}

%\extrainfo{\octocat~\url{http://github.com/wzpan}}

%\nopagenumbers{}                             % uncomment to suppress automatic page numbering for CVs longer than one page
%----------------------------------------------------------------------------------
%            content
%----------------------------------------------------------------------------------

\begin{document}
\hyphenpenalty=10000
\maketitle

\vspace{-3em}

\section{\hei 教育背景}

\tlcventry{2014}{2011}{\href{http://www.scnu.edu.cn}{华南师范大
    学}~\href{http://portal.scnu.edu.cn/wps/portal/cs}{计算机学院}}{研究生(保送)}{计算
  机应用技术}{}{}

\tlcventry{2011}{2007}{\href{http://www.scnu.edu.cn}{华南师范大
    学}~\href{http://portal.scnu.edu.cn/wps/portal/cs}{计算机学院}}{本科生}{网络
  工程}{}{}

\section{\hei 技能}

\cvcomputer{编程语言}{Python, C/C++, Java, Shell/Bash, ECMAScript}
           {操作系统}{Linux, Windows, Mac}
\cvcomputer{英语能力}{CET6(513)}
           {开发工具}{Emacs, Android Studio, Git, Makefile, gdb}
\cvcomputer{文档排版}{\LaTeX{}, Doxygen, Markdown}
           {测试工具}{minunit, nose, doctest}

\section{\hei 项目经验}

% Center labels and use "Since"
%\tltextstart[base]{\scriptsize}
%\tltextend[base]{\scriptsize}
%\tlsince{Since~}

\subsection{\hei 工作项目}

\tlcventry{2017}{0}{\href{http://tencent.com}{腾讯公司}}{高级工程师}{}{}
{
  \begin{tightitemize}%
  \item 腾讯公司SNG社交平台部应用开发组员工,参与旗下智能音箱
    \href{https://qrobot.qq.com/}{小Q机器人第二代}客户端及后台研发工作,以及一个新的教育项目的客户端及后台研发工作。
  \item 负责连网重构项目:重构机器人的网络连接功能,深入 Android 系统源码,
    实现各种异常网络状态的捕获和通知。
  \item 负责无意图闲聊优化项目:基于TF-IDF $+$ 余弦相似度统计出一代机器人的 30,000 组常见问题,
    并开发优化平台,联合竞品生成高频问题的回答,提供人工标定界面,并自动生
    成人工干预词条。
  \item 负责 APP 升级后台项目:支持应用动态更新升级。
  \item 试用期期间,申请专利 2 项,组织技术分享 1 次,发表公司级别头条文章 1 篇。
  \end{tightitemize}}

\tlcventry{2017}{2015}{\href{http://pingan.com}{平安金融壹账通}}{高级开发工程
  师}{}{}
{
  \begin{tightitemize}%
  \item Hyperion跨平台App开发框架核心成员;
  \item 设计开发了React-Native热更新、H5离线缓存框架、三合一更新器等,提升应用动
    态化能力;
  \item 开发团队代码管理工具
    \href{http://hahack.com/work/enterprise-class-git-version-control-1/}{fmanager}
    ,精通团队代码管理,擅长将复杂的代码管理任务\href{http://hahack.com/tags/Git/}{化繁为简};
  \item 业余时间为团队开发了微信机器人管家、Code Review平台、代码模块可视化
    平台、任意门线上日志分析平台、全自动加班信息统计平台、“百宝箱”内部网盘、
    “星黎殿”部门书签、“壹翎阁”技术论坛等,大幅提升团队效率。
  \item 平安金融壹账通2016年年度优秀员工,平安科技2015年Q3季度杰出贡献奖;
  \item 项目奖 $\times 1$,工具奖 $\times 2$, 创新奖 $\times 4$,申请专利 8 项
    ;
  \item 人工智能兴趣小组负责人,组织多次机器学习相关技术分享,Udacity 课程 Follower;
  \item 担任 3A 技术论坛演讲嘉宾;
  \item 工作期间保持绩效评级 A 等 3 次。
 \end{tightitemize}}

\tlcventry{2015}{2014}{\href{http://baidu.com}{百度国际科技(深圳)有限公司}}{开
  发测试工程师}{}{}
{
  \begin{tightitemize}%
  \item 移动 APP 源码 crash 隐患扫描工具 \href{http://godeyes.duapp.com}{GodEyes-iOS} 开发者(用户包括:百度、阿里、腾讯、京东);
  \item 开发设计路径规划 BadCase 挖掘平台,挖掘千余条不合理规划路径,提升产品路
    径规划能力;
  \item 开发一系列底图格式质检工具工具、svn监控工具、
    流量分析工具等工具成为了图像组的基础测试工具。
 \end{tightitemize}}

\subsection{\hei 科研项目}

\tlcventry{2013}{2009}{\href{http://www.siat.ac.cn/}{中国科学院深圳可视计算与可视分析重点实验室}}{实习生}{}{}
{
  \begin{tightitemize}%
  \item 基于模型驱动和神经网络的材质演变算法;
  \item 基于二维快速傅里叶变换的形状检索;
  \item 基于k-means算法的三维运动数据聚类(代码已\href{https://github.com/wzpan/MotionSegmentation3D}{开源});
  \item 为一个街景导航系统开发了二维地图控件和离线地图数据加密模块;
  \item 5篇核心期刊论文。1项发明专利。中国科学院客座学生奖学金。
\end{tightitemize}}

\newpage

\subsection{\hei 技术社区}

\tlcventry{2014}{2011}{\href{http://www.gzlug.org/}{广州 GNU/Linux用户组}}{活跃成员}{}{}
{
  \begin{tightitemize}%
  \item 举办 2011 年软件自由日(Software Freedom Day)广州地区庆祝活动并做技术分享;
  \item 常年参与线上技术交流活动及线下 Hacking Thursday 活动。
  \end{tightitemize}}

\tlcventry{2017}{}{\href{http://hexo.io}{Hexo}}{维护者}{}{}
{
  \begin{tightitemize}%
  \item 为流行的个人博客引擎 Hexo 开发了一系列渲染器、检索源生成器、Tag 插件和主
    题。
  \item 2017年受邀正式成为 hexojs 官方 Github 组织成员。
  \end{tightitemize}}

\tlcventry{2014}{2013}{\href{http://bbs.chinatex.org}{ChinaTeX}}{论坛精英}{}{}
{
  \begin{tightitemize}%
  \item 在论坛分享了多篇 \LaTeX{} 的学习心得。
  \item 高校 \LaTeX{} 推广者。开发华南师范大学的硕士生学位 \LaTeX{} 模板
    \href{https://github.com/scnu/scnuthesis}{SCNUThesis},支持在线编写,并被广东工业大学、首都师范大学等高校引用。
  \end{tightitemize}}

\subsection{\hei 个人开源项目}

\tlcventry{2017}{}{\href{http://github.com/wzpan/dingdang-robot}{dingdang-robot}}{作者}{}{}%
  {
\begin{itemize}
\item \textbf{项目简述}:类似 Amazon Echo 的机器人叮当。能工作在 Raspberry Pi 上
  的智能对话机器人,目的是提供一个适用于中文环境的开箱即用的中文对话机器人;
\item \textbf{主要特性}:支持选择离线唤醒SST引擎、在线SST引擎和TTS引擎,支持
  接入聊天机器人,支持接入微信,无缝联动 HomeAssistant 。支持插件模块化,容易拓展。
\item \textbf{项目指数}:460 stars,140 forks。QQ 用户群 352 名。技能插件 24 个。
\end{itemize}}

\tlcventry{2014}{2013}{\href{http://github.com/wzpan/qtevm}{QtEVM}}{作者}{}{}%
  {
\begin{itemize}
\item \textbf{项目简述}:欧拉影像放大技术(Eulerian Video Magnification)的开源
  实现;
\item \textbf{主要特性}:首个完整的 C++ 开源实现,能同时放大动作变化和颜色变化;
\item \textbf{项目指数}:53 stars,29 forks。
\end{itemize}}

\tlcventry{2013}{}{\href{http://github.com/wzpan/hexo-theme-freemind}{hexo-theme-freemind}}{作者}{}{}%
  {
\begin{itemize}
\item \textbf{项目简述}:最受欢迎的 Hexo 博客主题之一;
\item \textbf{主要特性}:丰富的 tag 插件,多种颜色主题,本地搜索引擎;
\item \textbf{项目指数}:272 stars,122 forks。
\end{itemize}}

\tlcventry{2013}{}{\href{http://github.com/wzpan/hexo-generator-search}{hexo-generator-search}}{作者}{}{}%
  {
\begin{itemize}
\item \textbf{项目简述}:一个能生成 Hexo 站点检索数据的插件;
\item \textbf{主要特性}:同时支持生成 XML 和
  JSON 两种格式,支持定制生成的内容范围;
\item \textbf{项目指数}:103 stars,18 forks。被多个主题内置。NPM 月均下载量 2000 次。
\end{itemize}}

\section{\hei 主要奖项}

% Restore normal labels
%\tltext{\scriptsize}

{

\tldatelabelcventry{2016}{2016.12}{优秀员工奖}{平安金融壹账通}{}{}{}

\setlength{\parskip}{-20pt}

  
\tldatelabelcventry{2015}{2015.10}{杰出贡献奖}{平安金融壹账通}{}{}{}

\setlength{\parskip}{-20pt}

\tldatelabelcventry{2015}{2015.12}{“一狼当先”创新员工奖}{平安金融壹账通}{}{}{}

\setlength{\parskip}{-20pt}


\tldatelabelcventry{2015}{2015.10}{双月创新奖}{平安金融壹账通}{}{}{}

\setlength{\parskip}{-20pt}

\tldatelabelcventry{2013}{2013.10}{国家奖学金}{国家级}{}{}{}

\setlength{\parskip}{-20pt}
  
\tldatelabelcventry{2011}{2011.10}{第十二届“挑战杯”全国大学生课外学术科技
  作品竞赛银奖}{国家级}{}{}{}

\tldatelabelcventry{2011}{2011.6}{优秀毕业生,优秀毕业论文}{校级}{}{}{}

\setlength{\parskip}{-10pt}

\tlcventry{2012}{2008}{校一等奖学金($\times 2$),校二等奖学金($\times 2$),创新奖学金}{校级}{}{}{}

}

\end{document}

%%% Local Variables:
%%% mode: latex
%%% TeX-master: t
%%% End:
