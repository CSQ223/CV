%% start of file `template.tex'.
%% Copyright 2006-2010 Xavier Danaux (xdanaux@gmail.com).
%
% This work may be distributed and/or modified under the
% conditions of the LaTeX Project Public License version 1.3c,
% available at http://www.latex-project.org/lppl/.

% Version: 20110122-4


\documentclass[11pt,a4paper,nolmodern]{moderncv}

\usepackage{WeizhouPan}

\usepackage[english]{babel}
\linespread{0.9}

\setCJKmainfont[BoldFont={Adobe Heiti Std},ItalicFont={Adobe Kaiti
  Std}]{Adobe Song Std}
\setCJKsansfont{YaHei Consolas Hybrid}
\setCJKmonofont{Adobe Kaiti Std}

\title{Weizhou Pan, 硕士研究生}
\address{华南师范大学~计算机学院}{}
\extrainfo{\octocat~\url{http://github.com/wzpan}\\
  \linkedin~\httplink{www.linkedin.com/in/hahack}%
}
%\myquote{\href{www.hahack.com}{blog: www.hahack.com}}{}

%\extrainfo{\octocat~\url{http://github.com/wzpan}}

%\nopagenumbers{}                             % uncomment to suppress automatic page numbering for CVs longer than one page
%----------------------------------------------------------------------------------
%            content
%----------------------------------------------------------------------------------

\begin{document}
\hyphenpenalty=10000
\maketitle

\vspace{-3em}


\section{\hei 教育背景}

\tlcventry{2011}{0}{\href{http://www.scnu.edu.cn}{华南师范大
    学}~\href{http://portal.scnu.edu.cn/wps/portal/cs}{计算机学院}}{研究生(保送)}{计算
  机应用}{}{}

\tlcventry{2007}{2011}{\href{http://www.scnu.edu.cn}{华南师范大
    学}~\href{http://portal.scnu.edu.cn/wps/portal/cs}{计算机学院}}{本科生}{网络
  工程}{}{}

\section{\hei 技能}

\cvcomputer{编程语言}{C/C++, Python, Shell/Bash}
           {操作系统}{Linux, Windows}
\cvcomputer{英语能力}{CET6(513)}
           {开发工具}{Emacs, Git, Makefile, gdb, Valgrind}
\cvcomputer{文档排版}{\LaTeX{}, Doxygen, Markdown, reST}
           {测试工具}{minunit, nose, doctest}
           
\section{\hei 项目经验}

% Center labels and use "Since"
%\tltextstart[base]{\scriptsize}
%\tltextend[base]{\scriptsize}
%\tlsince{Since~}

\subsection{\hei 科研项目}

\tlcventry{2012}{2013}{\href{http://www.siat.ac.cn/}{中国科学院深圳可视计算与可视分析重点实验室}}{实习生}{}{}
{
  \begin{tightitemize}%
  \item 基于模型驱动和神经网络的材质演变算法;
  \item 基于二维快速傅里叶变换的形状检索;
  \item 基于k-means算法的三维运动数据聚类(代码已开源);
  % \item 基于网格的图像变形算法(代码已开源);
  \item 为一个街景导航系统开发了二维地图控件和离线地图数据加密模块;
  \item 发表一篇核心期刊论文,获得中国科学院客座学生奖学金.
 \end{tightitemize}}

\tlcventry{2011}{2012}{虚拟手术场景}{核心成员}{}{}%
  {
\begin{itemize}
 \item \textbf{课题性质}:广东省产学研结合项目子课题.
 \item \textbf{研究内容}:搭建了一个虚拟现实手术环境:采用红外感应技术实现了一支仿真手术刀,并用基于BP
   神经网络的手势识别技术漫游场景;
 \item \textbf{个人职责}:研究红外感应技术,基于 BP 人工神经网络的手势识别;
 \item \textbf{项目成果}:两篇核心期刊论文;“挑战杯”科技学术竞赛全国银奖,广东省特
   等奖;
\end{itemize}}

\tlcventry{2009}{2010}{仿生歌唱系统}{负责人}{}{}%
  {
\begin{itemize}
\item \textbf{课题性质}:国家级大学生创新性实验计划项目;
\item \textbf{研究内容}: 设计了一种计算机学唱歌曲的系统:运用敲击定位法定
  位发音时刻,然后利用小波变换和快速傅里叶变换分析音调,最后采用语音合成技术输出声音;
\item \textbf{个人职责}:系统主要开发者,论文撰写人;
 \item \textbf{项目成果}:核心期刊论文.
\end{itemize}}

\subsection{\hei 开源项目}

\tlcventry{2012}{2013}{MotionSegmentation3D}{作者}{}{}%
  {
\begin{itemize}
\item \textbf{项目简述}:三维运动数据聚类函数库;
\item \textbf{主要特性}:将现有的二维运动数据聚类函数库扩展到三维,并对聚类结
  果进行优化;
\item \textbf{项目进展}:完成并开源.
\end{itemize}}

\tlcventry{2012}{2013}{SCNUThesis}{作者}{}{}%
  {
\begin{itemize}
\item \textbf{项目简述}:华南师范大学硕士/博士毕业论文LaTeX模板;
\item \textbf{主要特性}:提供多个模板选项,包括硕士/博士论文封面、单面/双面排版、TTF/OTF字库选择、盲评/非盲评论文等;
\item \textbf{项目进展}:维护阶段。累计下载量超过2,000次。有多名学生使用本模板完成了毕业论文排版.
\end{itemize}}

% \section{\hei 论文和专利}
% {
%   \tldatelabelcventry{2013}{2013.4}{沉浸式手术环境的研究与实现}{计算机仿
%     真}{第二作者}{已录用}{}
%   \setlength{\parskip}{-20pt}
%   \tldatelabelcventry{2013}{2013.2}{红外光虚拟手术刀的研究与实现}{计算机
%     系统应用}{第一作者}{已录用}{}
%   \tldatelabelcventry{2012}{2012.4}{一种基于小波和快速傅里叶变换的学习型歌唱系统
%   }{计算机工程与应用}{第一作者}{已发表}{}
%   \tldatelabelcventry{2013}{2013.8}{一种同步播放多媒体文件的方法及系统}{发
%     明}{第一申请人}{申请中}{}
%   \tldatelabelcventry{2012}{2012.12}{基于Clifford代数在十二方向上三维分割
%     腹部血管的应用实例}{发明}{合作申请人}{申请中}{}  
% }

\section{\hei 主要奖项}

% Restore normal labels
%\tltext{\scriptsize}

{

\tldatelabelcventry{2013}{2013.10}{国家奖学金}{国家级}{}{}{}

\setlength{\parskip}{-20pt}
  
\tldatelabelcventry{2011}{2011.10}{第十二届“挑战杯”全国大学生课外学术科技
  作品竞赛银奖}{国家级}{}{}{}

\tldatelabelcventry{2011}{2011.6}{优秀毕业生,优秀毕业论文}{校级}{}{}{}

\setlength{\parskip}{-10pt}

\tlcventry{2008}{2012}{校一等奖学金($\times 2$),校二等奖学金($\times 2$),创新奖学金}{校级}{}{}{}

%\tldatelabelcventry{2010}{2010.7}{华南师范大学二等奖学金 $\times$ 2}{校级}{}{}{}

}

\end{document}
